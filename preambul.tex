% Русификация
\usepackage[utf8x]{inputenc}
\usepackage[T2A]{fontenc}
\usepackage[english, russian]{babel}
% Параметры страницы
\usepackage{fullpage}
\usepackage{setspace}
\usepackage{geometry}
% Графика
\usepackage{chngcntr}
\usepackage{graphicx}
\usepackage{tikz}
% Таблицы
\usepackage{multirow}
\usepackage{tabularx}
\usepackage{longtable}
\usepackage{endlasthead}
\usepackage{tabu}
% Списки
\usepackage{enumitem}
% Формулы
\usepackage{amsmath}
% Листинг
\usepackage{color}
\usepackage{listings}
\usepackage{xcolor}
% Остальное
\usepackage[labelsep=endash,margin=10pt,justification=centerlast,format=hang,singlelinecheck=false]{caption}
%\usepackage{misccorr} % Баг: добавляет точки к номерам разделов
%\usepackage{indentfirst}
%\usepackage{fancynum}

% Tikz графика
\usetikzlibrary{arrows.meta,positioning}
\usetikzlibrary{shapes,snakes}
\tikzset{
	block/.style={inner sep=0mm},
	string/.style={align=center,inner sep=1em},
	text-block/.style={string,text width=4cm, minimum height=1cm,font=\large},
	triangle/.style={regular polygon,regular polygon sides=3},
	block-r/.style={block,rectangle,draw=black},
	block-c/.style={block,circle,draw=black},
	block-t/.style={block,triangle,draw=black},
	rect/.style={block-r,text-block,font=\normalfont},
}

% Параметры страницы
\geometry{left=3cm,right=1cm,top=2cm,bottom=2cm} % Поля
\linespread{1} % Интервал (1.6 - двойной, 1.3 - полуторный, 1 - одинарный)
\setlength\parindent{1.25cm} % Абзацный отступ

% Русификация (babel)
\addto\extrasrussian{%
	\renewcommand\figurename{Рис.}
	\renewcommand\contentsname{Содержание}
}
% Стилизация английского текста
\newcommand{\en}[1]{\emph{#1}}

% Шрифт TIMES NEW ROMAN
\renewcommand{\rmdefault}{ftm}

% ОГЛАВЛЕНИЕ
\makeatletter
\renewcommand{\tableofcontents}{\section*{\centering\contentsname}\@starttoc{toc}}
\setcounter{tocdepth}{2} % Глубина содержания
% dottedtocline параметры: 1) глубина вложенности, 2) отступ слева, 3) место под номер страницы 
\renewcommand{\l@section}{\@dottedtocline{1}{0cm}{2.3em}} % Стиль для секции. 
\renewcommand{\l@subsection}{\@dottedtocline{2}{1.25cm}{2.3em}} % Стилья для подсекции
\renewcommand{\@dotsep}{4} % расстояние между точками при заполнении
\makeatother

% Нумирация рисунков, таблиц и формул
\makeatletter
\counterwithin{figure}{section}
\counterwithin{table}{section}
\counterwithin{equation}{section}
\renewcommand{\thefigure}{\thesection.\@arabic\c@figure}
\renewcommand{\thetable}{\thesection.\@arabic\c@table}
\makeatother

%СЕКЦИИ БЕЗ НОМЕРА И ПО ЦЕНТРУ
\newcommand{\csection}[1]{
	\section*{\centering #1}
	\addcontentsline{toc}{section}{#1}
}

%РАЗДЕЛЫ
\makeatletter
\renewcommand{\section}{\@startsection{section}{1}%
	{\parindent} % Задает абзацный отступ
	{3.5ex plus 1ex minus .2ex} % вертикальный отступ перед 
	{2.3ex plus.2ex} % вертикальный отступ после 
	{\normalfont\Large}}
\renewcommand{\subsection}{\@startsection{subsection}{1}%
	{\parindent} % Задает абзацный отступ
	{3.5ex plus 1ex minus .2ex} % вертикальный отступ перед 
	{2.3ex plus.2ex} % вертикальный отступ после 
	{\normalfont\large}}
\makeatother

%ТАБЛИЦЫ

%СПИСКИ
\makeatletter
\AddEnumerateCounter{\asbuk}{\@asbuk}{м)}
\makeatother
\setlist{nolistsep,itemindent=\parindent,leftmargin=\parindent} % стили для всех списков
\setlist[1]{leftmargin=0\parindent} % стили для списков первого уровня
\setlist[enumerate,1]{label=\arabic{enumi})}
\setlist[enumerate,2]{label=\asbuk{enumii})}
\setlist[enumerate,3]{label=\alph{enumiii})}
\setlist[itemize,1]{label=--}

\binoppenalty=10000 % Запретить разрывы строк в формулах
%\relpenalty=10000 % тоже самое

% Подавление висячих строк
\clubpenalty=10000 
\widowpenalty=10000

\exhyphenpenalty=10000
\doublehyphendemerits=10000
\finalhyphendemerits=5000
\sloppy

% Математические функции
\newcommand\krugloed\partial

% ОФОРМЛЕНИЕ ЛИСТИНГА

\DeclareCaptionFormat{listing}{\parbox{\textwidth}{#1. #3}}
\captionsetup[lstlisting]{format=listing}

\definecolor{codegreen}{rgb}{0,0.6,0}
\definecolor{codegray}{rgb}{0.5,0.5,0.5}
\definecolor{codepurple}{rgb}{0.58,0,0.82}
\definecolor{backcolour}{rgb}{0.98,0.98,0.98}

\newcommand{\tracking}[2]{#2}

\lstdefinestyle{mystyle}{
	backgroundcolor=\color{backcolour},
	commentstyle=\color{codegreen},
	keywordstyle=\color{blue},
	numberstyle=\tiny\color{codegray},
	stringstyle=\color{codepurple},
	basicstyle=\footnotesize,
	breakatwhitespace=false,
	breaklines=true,
	captionpos=t,
	keepspaces=true,
	numbers=left,
	numbersep=10pt,
	showspaces=false,
	showstringspaces=false
	showtabs=false,
	tabsize=2,
	frame=tb
}

\lstset{style=mystyle}

% Цвета для кода

\definecolor{string}{HTML}{B40000} % цвет строк в коде
\definecolor{comment}{HTML}{008000} % цвет комментариев в коде
\definecolor{keyword}{HTML}{1A00FF} % цвет ключевых слов в коде
\definecolor{morecomment}{HTML}{8000FF} % цвет include и других элементов в коде
\definecolor{сaptiontext}{HTML}{FFFFFF} % цвет текста заголовка в коде
\definecolor{сaptionbk}{HTML}{999999} % цвет фона заголовка в коде
\definecolor{bk}{HTML}{FFFFFF} % цвет фона в коде
\definecolor{frame}{HTML}{999999} % цвет рамки в коде
\definecolor{brackets}{HTML}{B40000} % цвет скобок в коде

%%% Отображение кода %%%

% Настройки отображения кода

\lstset{
	%morekeywords={*,...}, % если хотите добавить ключевые слова, то добавляйте	 
	% Настройки отображения     
	breaklines=false, % Перенос длинных строк
	% Для отображения русского языка
	extendedchars=true,
	literate={Ö}{{\"O}}1
	{Ä}{{\"A}}1
	{Ü}{{\"U}}1
	{ß}{{\ss}}1
	{ü}{{\"u}}1
	{ä}{{\"a}}1
	{ö}{{\"o}}1
	{~}{{\textasciitilde}}1
	{а}{{\selectfont\char224}}1
	{б}{{\selectfont\char225}}1
	{в}{{\selectfont\char226}}1
	{г}{{\selectfont\char227}}1
	{д}{{\selectfont\char228}}1
	{е}{{\selectfont\char229}}1
	{ё}{{\"e}}1
	{ж}{{\selectfont\char230}}1
	{з}{{\selectfont\char231}}1
	{и}{{\selectfont\char232}}1
	{й}{{\selectfont\char233}}1
	{к}{{\selectfont\char234}}1
	{л}{{\selectfont\char235}}1
	{м}{{\selectfont\char236}}1
	{н}{{\selectfont\char237}}1
	{о}{{\selectfont\char238}}1
	{п}{{\selectfont\char239}}1
	{р}{{\selectfont\char240}}1
	{с}{{\selectfont\char241}}1
	{т}{{\selectfont\char242}}1
	{у}{{\selectfont\char243}}1
	{ф}{{\selectfont\char244}}1
	{х}{{\selectfont\char245}}1
	{ц}{{\selectfont\char246}}1
	{ч}{{\selectfont\char247}}1
	{ш}{{\selectfont\char248}}1
	{щ}{{\selectfont\char249}}1
	{ъ}{{\selectfont\char250}}1
	{ы}{{\selectfont\char251}}1
	{ь}{{\selectfont\char252}}1
	{э}{{\selectfont\char253}}1
	{ю}{{\selectfont\char254}}1
	{я}{{\selectfont\char255}}1
	{А}{{\selectfont\char192}}1
	{Б}{{\selectfont\char193}}1
	{В}{{\selectfont\char194}}1
	{Г}{{\selectfont\char195}}1
	{Д}{{\selectfont\char196}}1
	{Е}{{\selectfont\char197}}1
	{Ё}{{\"E}}1
	{Ж}{{\selectfont\char198}}1
	{З}{{\selectfont\char199}}1
	{И}{{\selectfont\char200}}1
	{Й}{{\selectfont\char201}}1
	{К}{{\selectfont\char202}}1
	{Л}{{\selectfont\char203}}1
	{М}{{\selectfont\char204}}1
	{Н}{{\selectfont\char205}}1
	{О}{{\selectfont\char206}}1
	{П}{{\selectfont\char207}}1
	{Р}{{\selectfont\char208}}1
	{С}{{\selectfont\char209}}1
	{Т}{{\selectfont\char210}}1
	{У}{{\selectfont\char211}}1
	{Ф}{{\selectfont\char212}}1
	{Х}{{\selectfont\char213}}1
	{Ц}{{\selectfont\char214}}1
	{Ч}{{\selectfont\char215}}1
	{Ш}{{\selectfont\char216}}1
	{Щ}{{\selectfont\char217}}1
	{Ъ}{{\selectfont\char218}}1
	{Ы}{{\selectfont\char219}}1
	{Ь}{{\selectfont\char220}}1
	{Э}{{\selectfont\char221}}1
	{Ю}{{\selectfont\char222}}1
	{Я}{{\selectfont\char223}}1
	{і}{{\selectfont\char105}}1
	{ї}{{\selectfont\char168}}1
	{є}{{\selectfont\char185}}1
	{ґ}{{\selectfont\char160}}1
	{І}{{\selectfont\char73}}1
	{Ї}{{\selectfont\char136}}1
	{Є}{{\selectfont\char153}}1
	{Ґ}{{\selectfont\char128}}1
	{\{}{{{\color{brackets}\{}}}1 % Цвет скобок {
	{\}}{{{\color{brackets}\}}}}1 % Цвет скобок }
}